\documentclass[a4paper,12pt]{article}
\usepackage[a4paper,margin=1in]{geometry}
\usepackage{adjustbox}
\setlength{\parindent}{0pt}
\begin{document}
	\begin{center}
		\section*{Computer Science II \\ Spring 2017}
	\end{center}
	Instructor: Dr.~Turner \hfill Email: \verb+dturner@csusb.edu+ \\
	Teaching Assistant: Mark Swoope \hfill Email: \verb+markswoope0@gmail.com+ \\
	\subsection*{Course Webpage}
	\verb+https://protovision.github.io/202-2017-spring/+
	\subsection*{Objective}
	To learn the eseential concepts and standard library components of the C++ programming language.
	\subsection*{Prerequisites}
	Computer Science I (CSE 201)
	\subsection*{Course Schedule}
	\begin{adjustbox}{width=1\textwidth}
	\begin{tabular}{|l|l|l|}
		\hline
		Date & Lecture & Assignment \\
		\hline
		04\slash 07\slash 2017 & Standard streams and I/O manipulators & Cloud 9 setup and filter programs \\
		\hline
		04\slash 14\slash 2017 & Vectors and Maps & Calculation programs \\
		\hline
		04\slash 21\slash 2017 & Arrays, pointers, and command-line arguments & Exploitation \\
		\hline
		04\slash 28\slash 2017 & File streams and string streams & Programs that can remember \\
		\hline
		05\slash 5\slash 2017 & Stacks and queues & Text parsing \\
		\hline
		05\slash 12\slash 2017 & Pseudo-random number generation & Games of chance \\
		\hline
		05\slash 19\slash 2017 & Constructors and initialization & TBA \\
		\hline
		05\slash 26\slash 2017 & Derived classes and virtual functions & TBA \\
		\hline
		06\slash 02\slash 2017 & Conversions and operator overloading & TBA \\
		\hline
		06\slash 09\slash 2017 & Review & -- \\
		\hline
		06\slash 16\slash 2017 & -- & Final project \\
		\hline
	\end{tabular}
	\end{adjustbox}
	\subsection*{Assignment submissions}
	Assignments are performed inside the Cloud 9 website. Your work is automatically checked each week.
	\subsection*{Grading policy}
	Your final grade is composed from 72\% normal assignments and 28\% final assignment. Assignments completed late will not be accepted. Identical assignments and plagiarized assignments will not be accepted. \\
	
	The letter for your final grade is based on these percentage ranges: 95-100 (A), 90-94 (A-), 87-89 (B+), 84-86 (B-), 80-83 (B), 75-79 (C+), 71-74 (C), 70-73 (C-), 67-69 (D+), 64-66 (D), 60-63 (D-), 0-59 (F). \\
	\newpage
	\subsection*{Additional information}
	\subsubsection*{Learning Outcomes}
	This course is designed to contribute to the following learning outcomes:
	\begin{itemize}
		\item An ability to apply knowledge of computing and mathematics appropriate to the discipline.
		\item An ability to analyze a problem, and identify and define the computing requirements appropriate to its solution.
		\item An ability to design, implement, and evaluate a computer-based system, process, component, or program to meet desired needs.
		\item An ability to use current techniques, skills, and tools necessary for computing practice.
		\item An ability to apply design and development principles in the contruction of software systems of varying complexity.
	\end{itemize}
	\section*{Students with disabilities}
	{
	\large
	If you are in need of an accommodation for a disability in order to participate in this class, please let us know as soon as possible, and also contact Services to Students with Disabilities at UH-183, \mbox{(909) 537-5238}. You are advised to establish a buddy system and alternate in the class if you require assistance in the event of an emergency. Individuals with disabilities should prepare for an emergency ahead of time by instructing a classmate and the instructor.
	}
	\subsubsection*{Academic Regulations and Procedures}
	See the CSUSB Bulletin of Courses for the University's policies on course withdrawal, cheating, and plagiarism.
\end{document}
